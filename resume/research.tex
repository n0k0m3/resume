%----------------------------------------------------------------------------------------
%	WORK EXPERIENCE SECTION
%----------------------------------------------------------------------------------------
\cvsection{Research Experience}

\begin{cventries}
    \cventry
    {Satellite 3D Reconstruction Research - Department of Mathematics and Systems Engineering}
    {Florida Tech}
    {Melbourne, FL}
    {Aug 2023 - Present}
    {\begin{cvitems}
            \item {Adapted Instant NeRF and D-NeRF algorithms for high-definition 3D modeling of resident space objects (RSOs) to assist with functionality identification and on-orbit servicing (OOS)}
            \item {Evaluated the algorithms for 3D reconstruction quality and hardware requirements using datasets of spacecraft mock-up images captured under various lighting and motion conditions at the Orbital Robotic Interaction, On-Orbit Servicing and Navigation (ORION) Laboratory}
            \item {Demonstrated the feasibility of training Instant NeRF on on-board computers to learn high-fidelity 3D models with manageable computational costs}
            \item {Developed an approach for mapping geometries of satellites on orbit based on 3D Gaussian splatting, capable of running on current spaceflight hardware}
            \item {Achieved nearly 2 orders of magnitude faster rendering of higher quality novel views of unknown satellites compared to previous NeRF-based algorithms, enabling on-board training and downstream machine intelligence tasks for autonomous guidance, navigation, and control}
            \item {Presented a novel approach for mapping geometries and high-confidence detection of components of unknown, non-cooperative satellites on orbit by combining accelerated 3D Gaussian splatting, virtual view rendering, and ensemble YOLOv5 object detection (submitted to CVPR 2024)}
        \end{cvitems}}

    \cventry
    {Suicide Prevention Research - Department of Computer Engineering and Sciences}
    {}
    {}
    {Jan 2023 - Jan 2024}
    {\begin{cvitems}
            \item {Enhanced data scraping pipelines for Twitter and Reddit, reducing ingestion time by 60 times using pure CLI tools (ripgrep, jq, awk, etc.) instead of databases like MySQL or MongoDB}
            \item {Employed a model explanation method, Layer Integrated Gradients, on top of a fine-tuned state-of-the-art encoder language model to assign attribution scores to tokens from Reddit users' posts for predicting suicidal ideation}
            \item {Proposed a methodology for preliminary screening of social media posts for suicidal ideation using the extracted token attributions, without relying on large language models during inference}
            \item {Developed a novel approach using off-the-shelf generative large language models (LLaMA2) to generate natural language explanations for suicide risk from users' Reddit posts}
            \item {Benchmarked various language models utilizing annotations and explanations by psychology experts, demonstrating the effectiveness of LLMs in classifying and responding with helpful reasoning for suicidal risk diagnosis}
            % \item {Optimized the method for low-resource settings by leveraging pre-existing general instruction-tuned and quantized models, enhancing accessibility and usability}
            % \item {Conducted user studies with experts to compare the explanations and predictions generated by the approach with human expert perception, providing valuable insights for further improvement}
            % \item {Analyzed monthly word statistics and word clouds of 800,000 suicidal posts over a 5-year period,  providing critical insights for suicide prevention efforts}
        \end{cvitems}}

    \cventry
    {Deep Reinforcement Learning for Robotics - Department Of Mechanical And Civil Engineering}
    {}
    {}
    {Jan 2022 - Sep 2022}
    {\begin{cvitems}
            \item {Investigated the feasibility of using transformers in off-policy Deep Reinforcement Learning problems, starting with Q-learning}
            \item {Explored the potential of transformers in solving Partially Observable Markov Decision Process (POMDP) problems}
            \item {Implemented a spatio-temporal transformer module and action-memory within Soft Actor-Critic (SAC) and Twin Delayed Deep Deterministic (TD3) policy gradient architectures}
            \item {Enabled the agent to "remember" its previous actions and effectively predict the next ones, enhancing decision-making capabilities}
            \item {Benchmarked the transformer-enhanced SAC and TD3 models against TD3-LSTM on \emph{highway-env}, an OpenAI Gym environment for autonomous driving decision-making tasks}
        \end{cvitems}}
\end{cventries}